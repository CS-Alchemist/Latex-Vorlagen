% ------------------------------------------------------------------------
% LaTeX - Preambel ******************************************************
% ------------------------------------------------------------------------
% Packages
% ------------------------------------------------------------------------
% basiernd auf www.matthiaspospiech.de/latex/vorlagen Diplomarbeit kompakt
% ========================================================================

% Inhalt:
% 1. Einige Pakete muessen unbedingt vor allen anderen geladen werden
% 2. Fonts Fonts Fonts
% 3. Math Packages
% 4. Symbole
% 5. text related packages
% 6. Pakete zum Zitieren
% 7. PDF related packages
% 8. Tables (Tabular)
% 9. figures and placement
% 10. verbatim packages
% 11. science packages
% 12. layout packages

% ~~~~~~~~~~~~~~~~~~~~~~~~~~~~~~~~~~~~~~~~~~~~~~~~~~~~~~~~~~~~~~~~~~~~~~~~
% Encoding der Dateien (sonst funktionieren Umlaute nicht)
% Empfohlen latin1, da einige Pakete mit utf8 Zeichen nicht
% funktionieren, z.B: listings, soul.

%\usepackage[latin1]{inputenx} % ISO-8859-1
%\usepackage[ansinew]{inputenx} % Windows-Standard (CP1252) (baut auf ISO 8859-1 und ISO 8859-15 auf)
\usepackage[utf8]{inputenc}

% ~~~~~~~~~~~~~~~~~~~~~~~~~~~~~~~~~~~~~~~~~~~~~~~~~~~~~~~~~~~~~~~~~~~~~~~~
% 1. Einige Pakete muessen unbedingt vor allen anderen geladen werden
% ~~~~~~~~~~~~~~~~~~~~~~~~~~~~~~~~~~~~~~~~~~~~~~~~~~~~~~~~~~~~~~~~~~~~~~~~
%
\usepackage{xspace} % Define commands that don't eat spaces.
\usepackage{ifpdf} % Fuer Pakete/Paketoptionen, die nur fuer pdf benoetigt werden \ifpdf \else \fi
\usepackage{calc} % Calculation with LaTeX
\usepackage[ngerman]{babel} % Languagesetting
\usepackage[table]{xcolor} % Farben
\usepackage[]{graphicx} % Bilder
%\usepackage{epstopdf} % If an eps image is detected, epstopdf is automatically called to convert it to pdf format.
\usepackage[]{amsmath} % Amsmath - Mathematik Basispaket
\usepackage{ragged2e} % Besserer Flatternsatz (Linksbuendig, statt Blocksatz)

% ~~~~~~~~~~~~~~~~~~~~~~~~~~~~~~~~~~~~~~~~~~~~~~~~~~~~~~~~~~~~~~~~~~~~~~~~
% 2. Fonts Fonts Fonts
% ~~~~~~~~~~~~~~~~~~~~~~~~~~~~~~~~~~~~~~~~~~~~~~~~~~~~~~~~~~~~~~~~~~~~~~~~

\usepackage[T1]{fontenc} % T1 Schrift Encoding (notwendig f�r die meisten Type 1 Schriften)
\usepackage{textcomp}	 % Zusatzliche Symbole (Text Companion font extension)

% Alle Schriften die hier angegeben sind sehen im PDF richtig aus.
% Die LaTeX Standardschrift ist die Latin Modern (lmodern Paket).
% If Latin Modern is not available for your distribution you must install the
% package cm-super instead. Otherwise your fonts will look horrible in the PDF

% DO NOT LOAD ae-Package for the font !

%% - Latin Modern
\usepackage{lmodern}
%% -------------------
%
% % - Times, Helvetica, Courier (Word Standard...)
%\usepackage{mathptmx}
%\usepackage[scaled=.90]{helvet}
%\usepackage{courier}
% % -------------------
%%
%% - Palantino , Helvetica, Courier
%\usepackage{mathpazo}
%\usepackage[scaled=.95]{helvet}
%\usepackage{courier}
%% -------------------
%
%% - Bera Schriften
%\usepackage{bera}
%% -------------------
%
%% - Charter, Bera Sans
%\usepackage{charter}\linespread{1.05}
%\renewcommand{\sfdefault}{fvs}


% ~~~~~~~~~~~~~~~~~~~~~~~~~~~~~~~~~~~~~~~~~~~~~~~~~~~~~~~~~~~~~~~~~~~~~~~~
% 3. Math Packages
% ~~~~~~~~~~~~~~~~~~~~~~~~~~~~~~~~~~~~~~~~~~~~~~~~~~~~~~~~~~~~~~~~~~~~~~~~

\usepackage[fixamsmath,disallowspaces]{mathtools} % Erweitert amsmath und behebt einige Bugs
\usepackage{fixmath}
\usepackage[all,warning]{onlyamsmath} % Warnt bei Benutzung von Befehlen die mit amsmath inkompatibel sind.
\usepackage{icomma} % Erlaubt die Benutzung von Kommas im Mathematikmodus

% ~~~~~~~~~~~~~~~~~~~~~~~~~~~~~~~~~~~~~~~~~~~~~~~~~~~~~~~~~~~~~~~~~~~~~~~~
% 4. Symbole
% ~~~~~~~~~~~~~~~~~~~~~~~~~~~~~~~~~~~~~~~~~~~~~~~~~~~~~~~~~~~~~~~~~~~~~~~~
\usepackage{amssymb}
%\usepackage{wasysym}
%\usepackage{marvosym}
%\usepackage{pifont}

% ~~~~~~~~~~~~~~~~~~~~~~~~~~~~~~~~~~~~~~~~~~~~~~~~~~~~~~~~~~~~~~~~~~~~~~~~
% 5. text related packages
% ~~~~~~~~~~~~~~~~~~~~~~~~~~~~~~~~~~~~~~~~~~~~~~~~~~~~~~~~~~~~~~~~~~~~~~~~

\usepackage{url} % Setzen von URLs. In Verbindung mit hyperref sind diese auch aktive Links.
\usepackage[stable,perpage, ragged,  multiple]{footmisc} % Fussnoten
\usepackage[ngerman]{varioref} % Intelligente Querverweise
\usepackage{enumitem} % Listen

% ~~~~~~~~~~~~~~~~~~~~~~~~~~~~~~~~~~~~~~~~~~~~~~~~~~~~~~~~~~~~~~~~~~~~~~~~
% 6. Pakete zum Zitieren
% ~~~~~~~~~~~~~~~~~~~~~~~~~~~~~~~~~~~~~~~~~~~~~~~~~~~~~~~~~~~~~~~~~~~~~~~~

\usepackage[babel, german=quotes, english=british, french=guillemets]{csquotes} % clever quotations
\SetBlockThreshold{2} % Anzahl von Zeilen
\newenvironment{myquote}%
          {\begin{quote}\small}%
          {\end{quote}}%
\SetBlockEnvironment{myquote}

% ~~~~~~~~~~~~~~~~~~~~~~~~~~~~~~~~~~~~~~~~~~~~~~~~~~~~~~~~~~~~~~~~~~~~~~~~
% 7. PDF related packages
% ~~~~~~~~~~~~~~~~~~~~~~~~~~~~~~~~~~~~~~~~~~~~~~~~~~~~~~~~~~~~~~~~~~~~~~~~

\ifpdf % Wenn als PDF ausgegeben wird
\usepackage{pdfpages} % pdf-Seiten einbinden
\usepackage[pdftex]{hyperref} % PDF Option in Hyperref
\else
\usepackage[dvipdfm]{hyperref}
\fi

%%% Doc: ftp://tug.ctan.org/pub/tex-archive/macros/latex/contrib/pdfpages/pdfpages.pdf
%\usepackage{pdfpages} % Include pages from external PDF documents in LaTeX documents

%%% Doc: ftp://tug.ctan.org/pub/tex-archive/macros/latex/contrib/hyperref/doc/manual.pdf
\hypersetup{
          pdfhighlight = /O,	         % Visualisierung beim anklicken von Links
% Farben fuer die Links
   colorlinks=true,	        % Links erhalten Farben statt Kaestchen
   urlcolor=darkblue,    % \href{...}{...} external (URL)
   filecolor=darkblue,  % \href{...} local file
   linkcolor=darkblue,  % \ref{...} and \pageref{...}
          citecolor =darkblue,    % Literaturverzeichnis
   % Links
   raiselinks=true,			 % calculate real height of the link
   breaklinks,	        % Links bestehen bei Zeilenumbruch
%   backref=page,	         % Backlinks im Literaturverzeichnis (section, slide, page, none)
%   pagebackref=true,        % Backlinks im Literaturverzeichnis mit Seitenangabe
   verbose,
%   hyperindex=true,         % backlinkex index
   linktocpage=true,        % Inhaltsverzeichnis verlinkt Seiten
%   hyperfootnotes=false,	% Keine Links auf Fussnoten
   % Bookmarks
%   bookmarks=true,	         % Erzeugung von Bookmarks fuer PDF-Viewer
   bookmarksopenlevel=1,    % Gliederungstiefe der Bookmarks
   bookmarksopen=true,      % Expandierte Untermenues in Bookmarks
   bookmarksnumbered=true,  % Nummerierung der Bookmarks
   bookmarkstype=toc,       % Art der Verzeichnisses
   % Anchors
   plainpages=false,        % % Make page anchors using the formatted form of the page number. With this option, hyperref writes different anchors for pages �ii� and �2�. (If the option is set �true� � the default � hyperref writes page anchors as the arabic form of the absolute page number, rather than the formatted form.)
   % hypertexnames=false,
   pageanchor=true,	        % Pages are linkable
   % PDF Informationen
   pdftitle={\workTyp: \workTitel},	        % Titel
   pdfauthor={\workNameStudent},	    % Autor
   pdfcreator={LaTeX, hyperref, KOMA-Script}, % Ersteller
   %pdfproducer={pdfeTeX 1.10b-2.1} %Produzent
   pdfstartview=FitH,       % Dokument wird Fit Width geaefnet
   pdfpagemode=UseOutlines, % Bookmarks im Viewer anzeigen
%   pdfpagelabels=true,      % set PDF page labels
}

% ~~~~~~~~~~~~~~~~~~~~~~~~~~~~~~~~~~~~~~~~~~~~~~~~~~~~~~~~~~~~~~~~~~~~~~~~
% 8. Tables (Tabular)
% ~~~~~~~~~~~~~~~~~~~~~~~~~~~~~~~~~~~~~~~~~~~~~~~~~~~~~~~~~~~~~~~~~~~~~~~~

\usepackage{booktabs}
\usepackage{tabularx} % tabularx nach hyperref laden
\usepackage{multirow}

% ~~~~~~~~~~~~~~~~~~~~~~~~~~~~~~~~~~~~~~~~~~~~~~~~~~~~~~~~~~~~~~~~~~~~~~~~
% 9. figures and placement
% ~~~~~~~~~~~~~~~~~~~~~~~~~~~~~~~~~~~~~~~~~~~~~~~~~~~~~~~~~~~~~~~~~~~~~~~~

%% Bilder und Graphiken ==================================================

\usepackage{float}	% Stellt die Option [H] fuer Floats zur Verfgung
\usepackage{flafter} % Floats immer erst nach der Referenz setzen
\usepackage{subfig} % Layout wird weiter unten festgelegt !
\usepackage{wrapfig} % Bilder von Text Umfliessen lassen

\usepackage{placeins} % Alle Floats bis \FloatBarrier ausgeben

% Make float placement easier
\renewcommand{\floatpagefraction}{.75} % vorher: .5
\renewcommand{\textfraction}{.1}       % vorher: .2
\renewcommand{\topfraction}{.8}        % vorher: .7
\renewcommand{\bottomfraction}{.5}     % vorher: .3
\setcounter{topnumber}{3}	         % vorher: 2
\setcounter{bottomnumber}{2}	         % vorher: 1
\setcounter{totalnumber}{5}	         % vorher: 3


% ~~~~~~~~~~~~~~~~~~~~~~~~~~~~~~~~~~~~~~~~~~~~~~~~~~~~~~~~~~~~~~~~~~~~~~~~
% 10. verbatim packages
% ~~~~~~~~~~~~~~~~~~~~~~~~~~~~~~~~~~~~~~~~~~~~~~~~~~~~~~~~~~~~~~~~~~~~~~~~

%%% Doc: ftp://tug.ctan.org/pub/tex-archive/macros/latex/contrib/upquote/upquote.sty
\usepackage{upquote} % Setzt "richtige" Quotes in verbatim-Umgebung

%%% Doc: No Documentation
% \usepackage{verbatim} % Reimplemntation of the original verbatim

%%% Doc: http://www.cs.brown.edu/system/software/latex/doc/fancyvrb.pdf
% \usepackage{fancyvrb} % Superior Verbatim Class

%% Listings Paket ------------------------------------------------------
%%% Doc: ftp://tug.ctan.org/pub/tex-archive/macros/latex/contrib/listings/listings-1.3.pdf
\usepackage{listings}

\lstset{
basicstyle =\ttfamily\color{black}\small, % Standardschrift
keywordstyle =, % \bfseries\color{blue}	  % Schl�sselwort-Style
%identifierstyle =\underbar,
commentstyle =\color{teal},
stringstyle =\itshape,
numbers = left,			  % Ort der Zeilennummern
numberstyle =\tiny\color{black},	   % Stil der Zeilennummern
numbers = left,			  % Ort der Zeilennummern
tabsize=2,			  % Groesse von Tabs
breaklines,			  % Zeilen werden Umgebrochen
breakatwhitespace,			  % An Leerzeichen umbrechen
%showspaces=true,			  % Leerzeichen anzeigen
backgroundcolor=\color{lightgray},	  % % Hintergrundfarbe der Listings
}
 \lstloadlanguages{% Check Dokumentation for further languages ...
%	[Visual]Basic
         [AlLaTeX]TeX,
         %Pascal
         %C
         %C++
         %XML
         %HTML
 }

%%% Doc: ftp://tug.ctan.org/pub/tex-archive/macros/latex/contrib/examplep/eurotex_2005_examplep.pdf
% LaTeX Code und Ergebnis nebeneinander darstellen
%\usepackage{examplep}


% ~~~~~~~~~~~~~~~~~~~~~~~~~~~~~~~~~~~~~~~~~~~~~~~~~~~~~~~~~~~~~~~~~~~~~~~~
% 11. science packages
% ~~~~~~~~~~~~~~~~~~~~~~~~~~~~~~~~~~~~~~~~~~~~~~~~~~~~~~~~~~~~~~~~~~~~~~~~

\usepackage[squaren]{SIunits}

% ~~~~~~~~~~~~~~~~~~~~~~~~~~~~~~~~~~~~~~~~~~~~~~~~~~~~~~~~~~~~~~~~~~~~~~~~
% 12. layout packages
% ~~~~~~~~~~~~~~~~~~~~~~~~~~~~~~~~~~~~~~~~~~~~~~~~~~~~~~~~~~~~~~~~~~~~~~~~

%% Zeilenabstand =========================================================
%
%%% Doc: ftp://tug.ctan.org/pub/tex-archive/macros/latex/contrib/setspace/setspace.sty
\usepackage{setspace}
%\doublespace	        % 2-facher Abstand
%\onehalfspace	  % 1,5-facher Abstand
% hereafter load 'typearea' again

%% Seitenlayout ==========================================================
%
% Layout mit 'typearea'
\typearea[current]{last}
\raggedbottom     % Variable Seitenhoehen zulassen


%% Kopf und Fusszeilen====================================================
%%% Doc: ftp://tug.ctan.org/pub/tex-archive/macros/latex/contrib/koma-script/scrguide.pdf
\usepackage[%
   automark,	 % automatische Aktualisierung der Kolumnentitel
   nouppercase,	 % Grossbuchstaben verhindern
]{scrlayer-scrpage}

\usepackage{scrtime} % Zeit
%\usepackage{scrdate} % Datum

\pagestyle{scrheadings} % Seite mit Headern
%\pagestyle{scrplain} % Seiten ohne Header
%\pagestyle{empty} % Seiten ohne Header

% loescht voreingestellte Stile
\clearscrheadings
\clearscrplain
%
% [scrplain]{scrheadings}

% %%% Kopfzeile
% einseitig: Bei einseitigem Layout, nur folgende Zeilen verwenden !!!
\ihead[]{\leftmark} % links: Kapitel
 %\chead[\pagemark]{\pagemark} % mitte:
\ohead[]{\rightmark} % rechts: Section

% %zweiseitig: Bei zweiseitigem Layout, nur folgende Zeilen verwenden !!!
%\ihead[]{} % innen
% % \chead[\pagemark]{\pagemark} % mitte:
%\ohead[]{\headmark} % aussen: Kapitel (linke Seite) und Section (rechte Seite)
%
% %%% Fusszeile
\ifoot[\workMarkDateTime]{\workMarkDateTime} % innen:
%\cfoot[\pagemark]{\pagemark} % mitte:
\ofoot[\pagemark]{\pagemark} % aussen: Seitenzahl

% Angezeigte Abschnitte im Header
\automark[section]{chapter} % Inhalt von [\rightmark]{\leftmark}
%
% Linie zwischen Kopf und Textk�rper
\setheadsepline{.4pt}[\color{black}]

%% Fussnoten =============================================================
% Keine hochgestellten Ziffern in der Fussnote (KOMA-Script-spezifisch):
\deffootnote{1.5em}{1em}{\makebox[1.5em][l]{\thefootnotemark}}
\addtolength{\skip\footins}{\baselineskip} % Abstand Text <-> Fussnote
\setlength{\dimen\footins}{10\baselineskip} % Beschraenkt den Platz von Fussnoten auf 10 Zeilen
\interfootnotelinepenalty=10000 % Verhindert das Fortsetzen von
                                % Fussnoten auf der gegen�berligenden Seite

%% Schriften (Sections )==================================================

% -- Koma Schriften --
\newcommand\SectionFontStyle{\sffamily}

\setkomafont{chapter}{\huge\SectionFontStyle}    % Chapter
\setkomafont{sectioning}{\SectionFontStyle} %  % Titelzeilen % \bfseries

\setkomafont{pagenumber}{\bfseries\SectionFontStyle} % Seitenzahl
\setkomafont{pagehead}{\small\sffamily}	       % Kopfzeile

\setkomafont{descriptionlabel}{\itshape}        % Stichwortliste
%
\renewcommand*{\raggedsection}{\raggedright} % Titelzeile linksbuendig, haengend
%

%% Captions (Schrift, Aussehen) ==========================================

%%% Doc: ftp://tug.ctan.org/pub/tex-archive/macros/latex/contrib/caption/caption.pdf
\usepackage{caption}
% Aussehen der Captions
\captionsetup{
   margin = 10pt,
   font = {small,rm},
   labelfont = {small,bf},
   format = plain, % oder 'hang'
   indention = 0em,	 % Einruecken der Beschriftung
   labelsep = colon, %period, space, quad, newline
   justification = RaggedRight, % justified, centering
   singlelinecheck = true, % false (true=bei einer Zeile immer zentrieren)
   position = bottom %top
}
%%% Bugfix Workaround
\DeclareCaptionOption{parskip}[]{}
\DeclareCaptionOption{parindent}[]{}

% Aussehen der Captions fuer subfigures (subfig-Paket)
\captionsetup[subfloat]{%
   margin = 10pt,
   font = {small,rm},
   labelfont = {small,bf},
   format = plain, % oder 'hang'
   indention = 0em,	 % Einruecken der Beschriftung
   labelsep = space, %period, space, quad, newline
   justification = RaggedRight, % justified, centering
   singlelinecheck = true, % false (true=bei einer Zeile immer zentrieren)
   position = bottom, %top
   labelformat = parens % simple, empty % Wie die Bezeichnung gesetzt wird
 }

%% Inhaltsverzeichnis (Schrift, Aussehen) sowie weitere Verzeichnisse ====

\setcounter{secnumdepth}{2}	 % Abbildungsnummerierung mit groesserer Tiefe
\setcounter{tocdepth}{2}		 % Inhaltsverzeichnis mit groesserer Tiefe
%

% Farben ================================================================
% Farben fuer die Links im PDF

\definecolor{green}{rgb}{0,0.5,0} % gr�n
\definecolor{brown}{rgb}{0.6,0,0} % braun
\definecolor{darkblue}{rgb}{0,0,.5} % dunkelblau
\definecolor{lightblue}{rgb}{0.8,0.85,1} % hellblau
% Farben fuer Listings
\colorlet{stringcolor}{green!40!black!100}
\colorlet{commencolor}{blue!0!black!100}


% Auszufuehrende Befehle  ------------------------------------------------

%\listfiles
%------------------------------------------------------------------------
